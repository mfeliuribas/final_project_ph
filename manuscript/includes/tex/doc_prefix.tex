\title{Language Attitudes of Catalan-Spanish Bilinguals Towards Intervocalic Alveolar Fricatives in Catalan}
\author{Please write XXXX instead of the name(s) of the author(s)}
\organization{Please write XXXX instead of the affiliation(s)}
\email{please write XXXX instead of the email address(es)}


\maketitle

\begin{abstract}
Catalan and Spanish are two languages that have coexisted for centuries in Eastern Spain. Thus, there is a bidirectional influence between these languages in all linguistic levels, including in the production of sounds. However, little is known about the perception and the attitudes toward a less Catalan-like production of a given sound. Therefore, the aim of this study is to examine the covert attitudes toward the production of intervocalic /s/, as either [s] or [z] by Catalan-Spanish bilinguals from Catalonia using a matched guise test. Results indicate that phoneme type is not a significant predictor of language attitudes in Catalonia, but other individual social factors, such as mother tongue and province, might be.
\end{abstract}

\keywords{Catalan, Spanish, language attitudes, intervocalic alveolar fricatives.}

